 here\documentclass[letterpaper,twocolumn,10pt]{article}
\usepackage{epsfig,xspace,url,authblk}


\title{Group Paper Review\\
Living IoT: A Flying Wireless Platform on Live Insects}
\author{Joseph Miera}
\author{Sam Maxwell}
\author{William Myers}
\affil{Brigham Young University}


\begin{document}

\maketitle

\subsection*{Introduction}

Put the introduction here. Write about where the paper came from, why it is important, etc.

\subsection*{Summary}

- Purpose of project: 
    - Create sensor network for irrigation and environmental sensing (especially for smart farming applications)
    - Sensor mobility significantly reduces overhead of manual sensor deployment
    - Explore idea of creating mobile wireless sensors using insects
- Why bees?
    - Use bees because mechanical drones don't have extended flight times due to lack of power (only last 5-30 min on single charge). 
    - Bees chemical reaction for energy is highly efficient and able to fly for extended periods of time.
    - Bees already common, many farms already buy and depend on bees to pollinate their crops
    - Bees can provide mobility for free
- What constraints does the system have to work under?
    - Small size and weight needed for bee to be able to carry it
    - Small size means VERY limited power
- How does the platform work?
    - Platform is called Living IoT
    - Backscattering sensor data - size, weight, and power requirements can only support very low processing and data rates
        - Common very low power method of sending data at lower rate
    - Sensor uses rf signals to localize themselves in 2D space
    - 900 MHz range has slower data rate than for example WiFi but offers longer range and uses less power
    - Envelope detector for self-localization
    - low-weight, low-power, small form-factor sensors (temperature, humidity, light)
    - Energy harvesting antenna to recharge battery overnight (when bees are in hive)
- Evaluation of system (How well does it work? What issues can there be?)
    - Multiple tests used to evaluate self-localization. Stationary and mobile tests
        - tests in soccer field (b/c flat and open). Localization worked up to 80m away.
        - tests in farm deployment
        - moving tests 


\subsection*{Strengths}

- Well-defined problem statement
    - Described their use case really well, and why what they were doing was useful
- Well-written introduction that clearly shows how they were able to solve the problem presented

\subsection*{Weaknesses}

- But what about the bees?
    - How much does this decrease the overall health and lifespan of the bee?
- 
- What was the cost in manual labor to deploy this?
    - How much specialization or training does it require?

\subsection*{Future Work}

- Increase backscatter range
- Controling insect flight
- Mass fabrication
- Camera sensing
- E-waste
- IACUC requirements and how they would apply to insects
- Developing drones simulating insect flight
- Solar powered sensors

\subsection*{Conclusion}

Put conclusion here. Write some closing thoughts and summarize the overall review of the paper.

\end{document}
